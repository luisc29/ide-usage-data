\chapter{Conclusions}

In this work we analyzed the impact of interruptions on the productivity and working patterns of developers as seen in usage data. This data came from two sources: the Usage Data Collector of Eclipse, with information of a variety of programmers, and a second dataset from Codealike that contains information of professional software developers from ABB.

Overall, we conclude that it is possible to observe in usage data the effects of events that occur in a working place (in our case interruptions) and the activities performed by developers. The results we obtained can be matched to observations from observational studies and experiments, so that we gained confidence in using this kind of information to understand complex phenomena. With that in mind, it may be possible in the near future to have better programming tools that learn from the activities of the user and offer assistance in common challenges and problems that developers face in a daily basis, and be able to mutate according to the circumstances. The results on work fragmentation covered in this thesis is under submission to the Journal of Software: Evolution and Process.

As future work, we consider valuable applying the same procedures while obtaining feedback from the developers. By doing so we can assure that our conclusions based on usage data are validated by the input from the developers. Also, we would be able to obtain more variables and enrich our results.

In Chapter 1 we defined a set of research questions as objectives grouped in two approaches: effect of interruptions and working patterns. Now we present a summary of our conclusions for each of the research questions. 

\textbf{RQ1: What is the relationship between the observed interruptions and the observed developer productivity?} 

We observed an inverse relationship between the number of interruptions and the productivity metrics. The editions per minute metric shows a stronger effect, specially in the first three groups involving from 0 to 3 interruptions. Beyond that point the effect seems smaller but still decreasing towards the groups with a high number of interruptions. This can be seen in the selections per minute as well and to a lesser extent in the edit ratio. We conclude that the number of interruptions has a negative effect on the productivity of developers, reducing the amount of activities performed.

\textbf{RQ2: Is the observed relationship more pronounced in the presence of prolonged interruptions?}

Then we investigated whether the length of the interruption has an increased effect, specifically if prolonged interruptions ($\geq$ 12 minutes) has a greater impact than shorter interruptions. Considering that the number of the interruptions has an effect, we created seven groups consisting of sessions with 1 to 7 interruptions and each was split into two additional groups according to the proportion of prolonged interruptions: low proportion and high proportion. We observed that the metrics show higher values when having a low number of interruptions and a low proportion of prolonged interruptions. So, we conclude that prolonged interruptions have a more disruptive effect than short interruptions.

\textbf{RQ3: What is the observed relationship in the vicinity of interruptions?}

Next, we look into the vicinity of interruptions and the behavior of the metrics 30 minutes before and after the interruption occurs. We observed an abrupt decrease of productivity in the immediate minutes around the interruption, approximately 10 minutes around it. Based on the literature, we are seeing the effect of the preparations before the interruption, and afterwards all the activities involved in the recovery of the context of the task. Moreover, we found three patterns of interruptions: positive, negative and neutral. This is an indication that not every interruption is disruptive, for some of them might be used to find solutions and answers.

\textbf{RQ4: What events are more common during recovery time?}

The last question involving interruptions is a follow up of the observations in the past research question. We observed a change of the metrics after the interruption, a phase called resumption or recovery time in the literature. Also we saw different patterns when having positive and negative interruptions. We delved deeper into the recovery time and analyzed which are the most common activities in this phase after a positive and negative interruption. We found that after a positive interruption there are more assertive actions like text edition and refactoring; we hypothesize that this is when the developer has the solution to a problem he faced before the interruption. On the other hand, after a negative interruption the actions are more related to program comprehension, like navigation around classes and debugging the program; we hypothesize that this occurs when the interruption has a disruptive effect the developer must perform the necessary activities to recover from it. We backed our findings with observations from the literature.

\textbf{RQ5: What kind of activities can be identified with interaction data?}
Having performed an analysis of work interruptions, we changed the subject to activities and working patterns and used the two available datasets. The first question in this context was meant to discover what kind of activities we can find in the data according to executed events. Using small time frames of activity, we executed clustering techniques to find them. In the UDC data we found eight activities and in the ABB data only six. Five activities appear in both datasets (Debugging, Programming, Navigation, Versioning and Tool usage) and the rest are unique. The percentages also differ between datasets and it can be due to the different tools used to capture them and the kind of programmers of the samples. Most of this activities have been characterized in observational studies and we lack information to detect activities related to collaboration between developers and interactions outside of the IDE.

\textbf{RQ6: What working patterns during a session are commonly performed by developers?}

Once we had all the activities performed by the developers, we split every working session into three phases and cluster them to find patterns of activities. We found several centroids (patterns) but the most common correspond to working sessions were the programmer mostly performs programming activities, program comprehension or general activities (multitasking). These three kind of activities can be observed in both datasets. This patterns of activities can be related to types of working sessions observed in observational studies.



